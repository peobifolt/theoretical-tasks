\documentclass{article}
\usepackage[utf8]{inputenc}
\usepackage[russian]{babel}
\usepackage{amsmath}
\usepackage{amssymb}
\usepackage{enumitem}
\usepackage{xcolor}
\usepackage[unicode, pdftex]{hyperref}
\usepackage[a4paper, left=2cm, right=2cm, top=2cm, bottom=2cm]{geometry}
\pagestyle{empty}

\newlist{tasks}{enumerate}{1}
\setlist[tasks,1]{label=\textcolor{blue}{\arabic*.}, left=0pt, itemsep=0pt}

\newcommand{\A}{\textcolor{green}{(a)}}
\newcommand{\B}{\textcolor{blue}{(b)}}
\newcommand{\todo}{\textcolor{red}{todo}}

\title{Теорсеминар №1}
\author{Кружки ИТМО. 21 октября 2024. 2 занятие, параллель A}
\date{}

\begin{document}
	
	\maketitle
	
	\href{https://docs.google.com/spreadsheets/d/1qnffTfPXIDFcwWoBrO4RRApMNMjeVCJ56XNfXEh3JYo/edit?usp=sharing}{\textcolor{blue}{Текущие результаты}}

	
	\begin{tasks}
		\item Дано дерево. За один ход игрок выбирает вершину и удаляет все ребра, выходящие из этой вершины. Проигрывает тот, кто не может сделать ход. Кто выиграет при оптимальной игре? $O(n)$.
		\item Дан массив целых чисел. Вес ребра между $i$-ой и $j$-ой вершинами равен $(a_i + a_j) \mod M$. Найдите минимальное остовное дерево. $O(n \log^2 n)$.
		\item Множество вершин графа называется выпуклым, если для любых двух вершин из этого множества все простые пути проходят только по этому множеству. Посчитайте по модулю $10^9 + 7$ количество выпуклых множеств в графе за $O(n + m)$.
		\item Турнир - это ориентированный граф, где между каждой парой вершин есть ровно одно ребро. Пусть $v < u$, если в турнире есть ребро $v \rightarrow u$. Докажите, что если запустить std::stable\_sort с таким компаратором, то он найдет гамильтонов путь.

		\item Дан неориентированный граф. Найдите цикл длины 4, или скажите, что такого нет за $O(n^2)$.
		\item Дан неориентированный граф. Найдите в нем путь длины 10, или скажите, что такого нет. $O(n + m)$.
		\item Даны запросы двух видов. Первый - провести ребро междву вершинами, гарантируется, что не появилось циклов. Второй - найти xor на пути между вершинами. $O(q \log n)$.
		\item Дан кактус и $q$ запросов. Первый вид - добавить вершину в множество. Второй вид - дана вершина, найти максимальное расстояние от этой вершины до вершины из множества. $O((n + q) \log n)$.
		\item Дан массив из $n$ целых чисел. Найдите подотрезок с минимальным $\max(a_{l...r}) \cdot AND(a_{l...r})$ за $O(n \log n)$, где AND - битовое И.
		\item За одну секунду фишка в ориентированном графе  перемещается равновероятно в одного из своих соседей. Посчитайте матемотическое ожидание попадания фишки из вершины $s$ в вершину $t$ за $O(n^3)$.	\end{tasks}
\end{document}
