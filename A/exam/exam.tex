\documentclass{article}
\usepackage[utf8]{inputenc}
\usepackage[russian]{babel}
\usepackage{amsmath}
\usepackage{amssymb}
\usepackage{enumitem}
\usepackage{xcolor}
\usepackage[unicode, pdftex]{hyperref}
\usepackage[a4paper, left=2cm, right=2cm, top=2cm, bottom=2cm]{geometry}
\pagestyle{empty}

\newlist{tasks}{enumerate}{1}
\setlist[tasks,1]{label=\textcolor{blue}{\arabic*.}, left=0pt, itemsep=0pt}

\newcommand{\A}{\textcolor{green}{(a)}}
\newcommand{\B}{\textcolor{blue}{(b)}}
\newcommand{\todo}{\textcolor{red}{todo}}



\title{Зимний зачет}
\author{Кружки ИТМО. 23 декабря 2024. Параллель A}
\date{}

\begin{document}
	
	\maketitle
	
	Изначально ваша оценка равна нулю. Вы можете её увеличить на один, решив любую задачу из соответствующего блока или две любые задачи из блоков, номера которых строго больше вашей текущей оценки. Каждую задачу можно решить максимум один раз.
	
	
	\begin{tasks}

		\item[1] Дана строка $s$. $lcp(i, j)$ - это длина наибольшего префикса строк $s_{i...}$ и $s_{j...}$. Найдите $lcp(i, j)$ для всех пар $i, j$ за $O(n^2)$.

		\item[2] Найдите количество пар $i < j$, что $a_i < a_j, b_i < b_j, c_i < c_j$ за $O(n^2/64)$.
		
		\item[3] Болото представленно в виде таблицы символов, в которой есть проходимые и непроходимые клетки. Вы стоите в середине таблицы, вы можете перейти в соседнюю проходимую клетку или перепрыгнуть через одну клетку, если после презимления вы окажитесь в проходимой клеке. Найдите минимальное количество прыжков, чтобы выбраться из болота за $O(nm)$.
		
		\item[4] Найдите количество совершенных паросочетаний в двудольном графе за $O(2^n n^2)$.

		\item[4] Даны строки $s$ и $t$. За одну операцию можно добавить символ в любую строку или удалить подстроку из любой строки. При этом, нельзя удалять всю строку целиком. Найти минимальное количество операций, чтобы сделать строки равными за $O(|s| |t|)$.
		
		
		\item[4] В доске $2$ на $n$ есть свободные и заблокированные клетки. Посчитайте количество способов замостить доску доминошками $2$ на $1$, $1$ на $2$ и плитками $1$ на $1$ за $O(n)$.
		
		
		\item[5] Дан массив размера $n$. Найти любое подмножество, сумма в котором делится на $n$ за $O(n)$.

		\item[5] Посчитайте $\sum_i \sum_j \sum_k a_i \oplus a_j \oplus a_k$ за $O(n \log C)$.

		\item[6] Посчитайте $\sum_{x = 1}^{n} x^2 \lfloor \frac{x}{n} \rfloor$ за $O(\sqrt n)$.

		\item[6] Есть $n \leq 10^6$ точек на плоскости, $0 \leq x_i, y_i \leq 10^6$. За один шаг можно перейти из одной точки в соседнюю. Постройте любой путь, который посещает все точки хотя бы по одному разу, чтобы его длина была не больше $2,5 \cdot 10^9$ за $O(n \log n)$.

		\item[7] Дана рекурента $a_n = \sum_{i=1}^{k} c_i a_{n - i}$. Найдите $a_0 + a_1 + ... + a_n$ за $O(k^3 \log n)$.

		\item[7] Дан массив чисел, $a_i \in \{-1, 1 \}$ и $q$ запросов. В запросе дан подотрезок и число $k$. На запрос надо ответить количество подотрезков отрезка из запроса, сумма на которых равна $k$. $O((n + q) \sqrt n)$.

		\item[8] Посчитайте за $O(n^2)$ количество перестановок $p$, для которых выполняются следующие условия. Вам дана строка $s$, $s_i \in \{ '<', '>' \}$. $i$-ое условие означает, что результат сравнения $p_i$ и $p_{i +1}$ такой же, какой символ записан в $s_i$.

		\item[8] Расскажите, как искать максимальную клику в графе за $o(2^n)$.

		\item[9] Решите рюкзак за $O(nW / 64)$ с восстановлением ответа.
		
		\item[9] Множество вершин называется независимым, если для любого ребра $(i, j)$ вершина $i$ не лежит в множестве или вершина $j$ не лежит в множестве. Найдите количество независимых множеств за $O(1,239^n)$.
		
		\item[10] Дан массив из $n$ целых неотрицательных чисел. Вес ребра из $i$ в $j$ равен сумме значений на интервате между $i$-ым и $j$-ым элементом, причем начальный элемент считается, а конечный нет. Путь, который проходит по всем вершинам ровно один раз называется новогодним, если его суммарный вес равен $S$. Посчитайте количество новогодних путей за $O(n^2 S)$.
		
		
		\item[10] Дано $0 \leq n \leq 10^9$. Посчитайте это по модулю $10^9 + 7$ за быстро.
		
		\[
		\left( \sum_{k=0}^{n-1} C_n^k \right) \cdot
		\left ( \sum_{k=0}^n C_n^k n^{k+1} \right) +
		\left( \sum_{k=0}^{n} (-1)^k C_n^k \right) \cdot
		\left( \sin n + e^n + \sum_{k=4}^{10^9} C_k^3) \right) +
		\sum_{k=1}^{\lfloor \frac{n}{2} \rfloor} C_n^{2k} + n +
		\sum_{k=0}^n C_n^k \cdot (k + 1) +
		\sum_{k=2}^n k^3
		\]

	\end{tasks}
\end{document}
